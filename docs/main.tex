\documentclass{article}
\usepackage{dot2texi}
\usepackage{tikz}
\usepackage{amsmath, amsfonts}
\usetikzlibrary{shapes,arrows}

%% ------------------------------------------------
%% headers for algorithm package
\usepackage{xcolor,amsmath,mathtools}
\usepackage[linesnumbered,
            ruled,
            vlined,
            noalgohanging,
            ]{algorithm2e}
\DontPrintSemicolon

% Define pseudocode formatting
\renewcommand{\KwSty}[1]{\textnormal{\textcolor{blue!90!black}{\ttfamily\bfseries #1}}\unskip}
\renewcommand{\ArgSty}[1]{\textnormal{\ttfamily #1}\unskip}
\SetKwComment{Comment}{\color{green!50!black}// }{}
\renewcommand{\CommentSty}[1]{\textnormal{\ttfamily\color{green!50!black}#1}\unskip}
\newcommand{\assign}{\leftarrow}
\newcommand{\var}{\texttt}
\newcommand{\FuncCall}[2]{\texttt{\bfseries #1(#2)}}
\SetKwProg{Function}{function}{}{}
\renewcommand{\ProgSty}[1]{\texttt{\bfseries #1}}

\DeclareMathOperator*{\argmax}{arg\,max}
\SetKwProg{Proc}{procedure}{}{}
%% ------------------------------------------------

\begin{document}

% \begin{dot2tex}[neato, mathmode]
%     digraph G  {
%         node [shape=plaintext];
%         p [label="+"];
%         t [texlbl="\LaTeX"];
%         r [texlbl="$\alpha_1$"];
%         10-> p;
%         6 -> t;
%         r -> t;
%         t -> p;
%     }
% \end{dot2tex}

\section{Why we're doing this!}

\section{Boundary conditions}

\begin{itemize}
    \item Isothermal ($\theta = 300 \ K$). The thermal driving force is therefore constant.
    \item $\mathbb C$ is constant since there is only a ferritic phase.
\end{itemize}

\section{Notation}
superscripts inside curley braces indicate the slip-system and should not be mistaken with exponents.
An equation without temporal subscript can be assumed to be valid in any timestep.
A second subscript, if exists, denotes the timestep or iteration number, depending on the context, i.e., ${\bf F}_{e, n+1}$ is the elastic deformation gradient at time step $n+1$.

\section[]{Summary of formula used in the numerical implementation}

\begin{equation}
    {\bf F}_{p, n+1} = \left({\bf I} - \sum_{i=1}^N \Delta \gamma^{(i)} {\bf m}^{(i)} \otimes {\bf n}^{(i)}\right)^{-1} {\bf F}_{p, n}
\end{equation}

\begin{equation}
    \bf S = \mathbb C \bf E
\end{equation}

\begin{equation}
    g_m^{(i)} = {\bf F}_e^{\sf T} {\bf F}_e {\bf S} \cdot \left( {\bf m}^{(i)} \otimes {\bf n}^{(i)} \right)
\end{equation}

\begin{equation}
    k^{(j)} = k_{F,0} \left( 1 - \frac{s^{(j)}}{s_{F,\infty}} \right)^{u_F}
\end{equation}

\begin{equation}
    H^{(i,j)} =
    \begin{cases}
        k^{(j)}       &   i = j\\
        q_F k^{(j)}   &   i \neq j
    \end{cases}
\end{equation}

\begin{equation}
    w^{(i)} = \frac{1}{c_F \mu_F N} \sum_{j=1}^N H^{(j,i)}
\end{equation}

\begin{equation}
    \beta_{n+1} = \beta_n + \Delta \beta = \beta_n + \sum_{i=1}^N w_{n+1}^{(i)} \Delta \gamma^{(i)}
\end{equation}

\begin{equation}
    g_d^{(i)} = \omega_F \mu_F \beta w^{(i)}
\end{equation}

\begin{equation}
    g_{th}^{(i)} = g_{th} = \rho_0 \theta \phi_F
\end{equation}

\begin{equation}
    g^{(i)} = g_{m}^{(i)} + g_{th}^{(i)} + g_{d}^{(i)}
\end{equation}


As before, we have two equations of rate form and use them for temporal discritization:

\begin{equation}
    \dot{s}^{(i)} = \sum_{j=1}^N H^{(i,j)} \dot{\gamma}^{(j)}
\end{equation}

\begin{equation}
    \dot{\gamma}^{(i)} =
    \begin{cases}
        \dot{\gamma}_{F,0} \left(\left(\frac{g^{(i)}}{s^{(i)}}\right)^{\frac{1}{p}} - 1 \right) & g^{(i)} > s^{(i)}\\
        0 & \text{otherwise}
    \end{cases}
\end{equation}

which lead to the following residuals

\begin{equation}
    r_I^{(i)} \coloneqq \Delta s^{(i)}\ - \sum_{j=1}^N H^{(i,j)} \Delta \gamma^{(j)}
\end{equation}


\begin{equation}
    r_{II}^{(i)} \coloneqq
    \begin{cases}
        \Delta {\gamma}^{(i)} - \Delta t \dot{\gamma}_0 & g^{(i)} > s^{(i)}\\
        0 & \text{otherwise}
    \end{cases}
\end{equation}


\section{Visualization of computed values}

The (forward) dependency graph of the residuals ${\bf r}_I$ and ${\bf r}_{II}$ is shown in figure \ref{fig:compu-graph} . Not shown in this graph are all the constants factors and values from timestep $n$, as well as field variables ${\bf F}_n$ and ${\bf F}_{n+1}$.

\begin{figure}[]
    % arrow type could be one of "-stealth, -to, -latex, -diamond, -o, ->>, -hooks"
\begin{dot2tex}[dot, mathmode, scale=0.8]
    digraph G{
        d2tfigpreamble = "\Huge";
        graph [mindist=0.5];
        edge [style="-stealth, very thick"];
        node [shape=none];

        k [texlbl="$k^{(i)}$"];
        H [texlbl="$H^{(i,j)}$"];
        w [texlbl="$w^{(i)}$"];
        beta [texlbl="$\beta$"];

        node [fixedsize=false, shape=none, width=1, style="fill=blue!20"];
        ds [texlbl="$\Delta s^{(i)}$"];
        dgamma [texlbl="$\Delta \gamma^{(i)}$"];
        node [shape=none, style=""];

        gm [texlbl="$g^{(i)}_m$"];
        gd [texlbl="$g^{(i)}_d$"];

        g [texlbl="$g^{(i)}$"];

        Fp [texlbl="${\bf F}_p$"];
        Fe [texlbl="${\bf F}_e$"];
        S [texlbl="${\bf S}$"];

        node [fixedsize=false, shape=none, width=1, style="fill=green!20"];
        r1 [texlbl="${\bf r}_I$"];
        r2 [texlbl="${\bf r}_{II}$"];
        node [shape=none, style=""];

        ds -> k;
        k -> H;
        H -> w;
        w -> beta;
        w -> gd;
        dgamma -> beta;
        beta -> gd;
        gd -> g;
        gm -> g;
        dgamma -> Fp;
        Fp -> Fe;
        Fe -> S;
        Fe -> gm;
        S -> gm;

        ds -> r1;
        dgamma -> r1;
        H -> r1;

        dgamma -> r2;
        g -> r2;
        ds -> r2;
    }
\end{dot2tex}

    \caption{Graph of forward call of computed residuals. The nodes that indicate roots are highlighted with pale blue, while green indicates the leaf of the graph.}
    \label{fig:compu-graph}
\end{figure}


\section{Algorithm}

\begin{algorithm}
    \caption{Return mapping algorithm with penalty term}
    \KwIn{\;
    \Indpp
    Iteration index: $k \assign 0$\;
    Primary variables:
    $\Delta {\boldsymbol \gamma}_k = {\bf 0}$,
    $\Delta {\bf s}_k = {\bf 0}$\;
    Converged variables from time step $n$: $\beta_n$,
    ${\bf F}_{p,n}$,
    $\Delta {\bf s}_n$\,
    $\Delta {\boldsymbol \gamma}_n$\;
    Field variables: ${\bf F}_n$, ${\bf F}_{n+1}$
	}

    \Proc{update} {
        \SetAlgoLined
        $k \assign k+1$\;
        update $\sigma_k$\;
    }
\end{algorithm}

Conditions $\Delta \gamma_k^{(i)} \ge 0$ and $\Delta s_{n+1}^{(i)} \le s_{\infty}$ translate to the following penalty term:

\begin{equation}
    \sigma_k \sum_{i=1}^N
    \left[
        \max \left( 0, - \Delta \gamma_k^{(i)}\right) ^ 2 +
        \max \left( 0, \Delta s_k^{(i)} + s_n^{(i)} - s_{\infty}\right) ^ 2
    \right]
\end{equation}

\subsection{Practical considerations}
\begin{itemize}
    \item How many iterations should be performed (maximum number of iterations)?
    \item Ratio of $\|{\bf r}_I\|$ and $\|{\bf r}_{II}\|$. Do we need to use normalization so they are comparable magnitude-wise? Report the residuals for diagnostic and debug.
    \item Report the rate of convergence for diagnostics
    \item When is it required to perform half stepping?
    \item Update $\sigma_k$ every iteration or every few iterations?
\end{itemize}


\end{document}
